\documentclass[aspectratio=169]{beamer}

\input{preamble.tex}

\subtitle{Bölüm 2: Yapılandırılmış Belgeler ve Daha Fazlası}

\begin{document}

%%%%%%%%%%%%%%%%%%%%%%%%%%%%%%%%%%%%%%%%%%%%%%%%%%%%%%%%%%%%%%%%%%%%%%%%%%%%%%%
%%%%%%%%%%%%%%%%%%%%%%%%%%%%%%%%%%%%%%%%%%%%%%%%%%%%%%%%%%%%%%%%%%%%%%%%%%%%%%%
%%%%%%%%%%%%%%%%%%%%%%%%%%%%%%%%%%%%%%%%%%%%%%%%%%%%%%%%%%%%%%%%%%%%%%%%%%%%%%%
\begin{frame}
\titlepage
\end{frame}

%%%%%%%%%%%%%%%%%%%%%%%%%%%%%%%%%%%%%%%%%%%%%%%%%%%%%%%%%%%%%%%%%%%%%%%%%%%%%%%
%%%%%%%%%%%%%%%%%%%%%%%%%%%%%%%%%%%%%%%%%%%%%%%%%%%%%%%%%%%%%%%%%%%%%%%%%%%%%%%
%%%%%%%%%%%%%%%%%%%%%%%%%%%%%%%%%%%%%%%%%%%%%%%%%%%%%%%%%%%%%%%%%%%%%%%%%%%%%%%
\section{Yapılandırılmış Belgeler}

%%%%%%%%%%%%%%%%%%%%%%%%%%%%%%%%%%%%%%%%%%%%%%%%%%%%%%%%%%%%%%%%%%%%%%%%%%%%%%%
%%%%%%%%%%%%%%%%%%%%%%%%%%%%%%%%%%%%%%%%%%%%%%%%%%%%%%%%%%%%%%%%%%%%%%%%%%%%%%%
%%%%%%%%%%%%%%%%%%%%%%%%%%%%%%%%%%%%%%%%%%%%%%%%%%%%%%%%%%%%%%%%%%%%%%%%%%%%%%%
\begin{frame}{Anahat}
\begin{multicols}{2}
\tableofcontents[currentsection]
\end{multicols}
\end{frame}

%%%%%%%%%%%%%%%%%%%%%%%%%%%%%%%%%%%%%%%%%%%%%%%%%%%%%%%%%%%%%%%%%%%%%%%%%%%%%%%
%%%%%%%%%%%%%%%%%%%%%%%%%%%%%%%%%%%%%%%%%%%%%%%%%%%%%%%%%%%%%%%%%%%%%%%%%%%%%%%
%%%%%%%%%%%%%%%%%%%%%%%%%%%%%%%%%%%%%%%%%%%%%%%%%%%%%%%%%%%%%%%%%%%%%%%%%%%%%%%
\begin{frame}{\insertsection}
\begin{itemize}
\item Bölüm 1'de, metin ve matematiği dizmek için komutlar ve ortamlar hakkında bilgi edindik.
\item Şimdi, belgeleri yapılandırmak için komutlar ve ortamlar hakkında bilgi edineceğiz.
\item Overleaf'ta yeni komutları deneyebilirsiniz:
\end{itemize}
\vskip 2em
\begin{center}
\fbox{\href{\wlnewdoc{basics.tex}}{%
Örnek belgeyi \wllogo{}'te açmak için burayı tıklayın}}
\\[1ex]\scriptsize{}
En iyi sonuçlar için lütfen \href{http://www.google.com/chrome}{Google Chrome} veya yeni bir \href{http://www.mozilla.org/en-US/firefox/new/}{FireFox} kullanın.
\end{center}
\vskip 2ex
\begin{itemize}
\item Başlayalım!
\end{itemize}
\end{frame}

%%%%%%%%%%%%%%%%%%%%%%%%%%%%%%%%%%%%%%%%%%%%%%%%%%%%%%%%%%%%%%%%%%%%%%%%%%%%%%%
%%%%%%%%%%%%%%%%%%%%%%%%%%%%%%%%%%%%%%%%%%%%%%%%%%%%%%%%%%%%%%%%%%%%%%%%%%%%%%%
%%%%%%%%%%%%%%%%%%%%%%%%%%%%%%%%%%%%%%%%%%%%%%%%%%%%%%%%%%%%%%%%%%%%%%%%%%%%%%%
\subsection{Başlık ve Özet}
\begin{frame}[fragile]{\insertsubsection}
\begin{itemize}{\small
\item Giriş kısmında \LaTeX{}'e \cmdbs{title} ve \cmdbs{author} isimlerini bildirin.
\item Ardından, başlığı gerçekten oluşturmak için belgede \cmdbs{maketitle} kullanın.
\item Bir özet yapmak için \bftt{abstract} ortamını kullanın.
}\end{itemize}
\begin{minipage}{0.55\linewidth}
\inputminted[fontsize=\scriptsize,frame=single,resetmargins]{latex}%
  {structure-title.tex}
\end{minipage}
\begin{minipage}{0.35\linewidth}
\includegraphics[width=\textwidth,clip,trim=2.2in 7in 2.2in 2in]{structure-title.pdf}
\end{minipage}
\end{frame}

%%%%%%%%%%%%%%%%%%%%%%%%%%%%%%%%%%%%%%%%%%%%%%%%%%%%%%%%%%%%%%%%%%%%%%%%%%%%%%%
%%%%%%%%%%%%%%%%%%%%%%%%%%%%%%%%%%%%%%%%%%%%%%%%%%%%%%%%%%%%%%%%%%%%%%%%%%%%%%%
%%%%%%%%%%%%%%%%%%%%%%%%%%%%%%%%%%%%%%%%%%%%%%%%%%%%%%%%%%%%%%%%%%%%%%%%%%%%%%%
\subsection{Bölümler}
\begin{frame}{\insertsubsection}
\begin{itemize}{\small
\item Sadece \cmdbs{section} ve \cmdbs{subsection} kullanın.
\item  \cmdbs{section*} ve \cmdbs{subsection*}ın ne işe yaradığını tahmin edebilir misiniz?
}\end{itemize}
\begin{minipage}{0.55\linewidth}
\inputminted[fontsize=\scriptsize,frame=single,resetmargins]{latex}%
  {structure-sections.tex}
\end{minipage}
\begin{minipage}{0.35\linewidth}
\includegraphics[width=\textwidth,clip,trim=1.5in 6in 4in 1in]{structure-sections.pdf}
\end{minipage}
\end{frame}

%%%%%%%%%%%%%%%%%%%%%%%%%%%%%%%%%%%%%%%%%%%%%%%%%%%%%%%%%%%%%%%%%%%%%%%%%%%%%%%
%%%%%%%%%%%%%%%%%%%%%%%%%%%%%%%%%%%%%%%%%%%%%%%%%%%%%%%%%%%%%%%%%%%%%%%%%%%%%%%
%%%%%%%%%%%%%%%%%%%%%%%%%%%%%%%%%%%%%%%%%%%%%%%%%%%%%%%%%%%%%%%%%%%%%%%%%%%%%%%
\subsection{Etiketler ve Çapraz Referanslar}
\begin{frame}[fragile]{\insertsubsection}
\begin{itemize}{\small
\item Otomatik numaralandırma için \cmdbs{label} ve \cmdbs{ref} kullanın.
\item \bftt{amsmath} paketi denklemleri referans almak için \cmdbs{eqref} kullanır
equations.
}\end{itemize}
\begin{minipage}{0.55\linewidth}
\inputminted[fontsize=\scriptsize,frame=single,resetmargins]{latex}%
  {structure-crossref.tex}
\end{minipage}
\begin{minipage}{0.35\linewidth}
\includegraphics[width=\textwidth,clip,trim=1.8in 6in 1.6in 1in]{structure-crossref.pdf}
\end{minipage}
\end{frame}

%%%%%%%%%%%%%%%%%%%%%%%%%%%%%%%%%%%%%%%%%%%%%%%%%%%%%%%%%%%%%%%%%%%%%%%%%%%%%%%
%%%%%%%%%%%%%%%%%%%%%%%%%%%%%%%%%%%%%%%%%%%%%%%%%%%%%%%%%%%%%%%%%%%%%%%%%%%%%%%
%%%%%%%%%%%%%%%%%%%%%%%%%%%%%%%%%%%%%%%%%%%%%%%%%%%%%%%%%%%%%%%%%%%%%%%%%%%%%%%
\subsection{Alıştırma}
\begin{frame}[fragile]{Yapılandırılmış Belge Alıştırması}

\begin{block}{Bu kısa makaleyi \LaTeX'de yazın:
\footnote{\url{http://pdos.csail.mit.edu/scigen/}, rastgele makale üreteci.}}
\begin{center}
\fbox{\href{\fileuri/structure-exercise-solution.pdf}{%
Makaleyi açmak için tıklayın}}
\end{center}
Kağıdını buna benzet. Bölüm ve denklem numaralarını metne açıkça yazmaktan kaçınmak için \cmdbs{ref} ve \cmdbs{eqref} kullanın.
\end{block}
\vskip 2ex
\begin{center}
\fbox{\href{\wlnewdoc{structure-exercise.tex}}{%
Alıştırmayı \wllogo{}'de açmak için tıklayın}}
\end{center}

\begin{itemize}
\item Denedikten sonra, çözümümü görmek için
\fbox{\href{\wlnewdoc{structure-exercise-solution.tex}}{%
 burayı tıklayın}}.
\end{itemize}
\end{frame}

%%%%%%%%%%%%%%%%%%%%%%%%%%%%%%%%%%%%%%%%%%%%%%%%%%%%%%%%%%%%%%%%%%%%%%%%%%%%%%%
%%%%%%%%%%%%%%%%%%%%%%%%%%%%%%%%%%%%%%%%%%%%%%%%%%%%%%%%%%%%%%%%%%%%%%%%%%%%%%%
%%%%%%%%%%%%%%%%%%%%%%%%%%%%%%%%%%%%%%%%%%%%%%%%%%%%%%%%%%%%%%%%%%%%%%%%%%%%%%%
\section{Şekiller ve Tablolar}

%%%%%%%%%%%%%%%%%%%%%%%%%%%%%%%%%%%%%%%%%%%%%%%%%%%%%%%%%%%%%%%%%%%%%%%%%%%%%%%
%%%%%%%%%%%%%%%%%%%%%%%%%%%%%%%%%%%%%%%%%%%%%%%%%%%%%%%%%%%%%%%%%%%%%%%%%%%%%%%
%%%%%%%%%%%%%%%%%%%%%%%%%%%%%%%%%%%%%%%%%%%%%%%%%%%%%%%%%%%%%%%%%%%%%%%%%%%%%%%
\begin{frame}{Anahat}
\begin{multicols}{2}
\tableofcontents[currentsection]
\end{multicols}
\end{frame}

%%%%%%%%%%%%%%%%%%%%%%%%%%%%%%%%%%%%%%%%%%%%%%%%%%%%%%%%%%%%%%%%%%%%%%%%%%%%%%%
%%%%%%%%%%%%%%%%%%%%%%%%%%%%%%%%%%%%%%%%%%%%%%%%%%%%%%%%%%%%%%%%%%%%%%%%%%%%%%%
%%%%%%%%%%%%%%%%%%%%%%%%%%%%%%%%%%%%%%%%%%%%%%%%%%%%%%%%%%%%%%%%%%%%%%%%%%%%%%%
\subsection{Grafikler}
\begin{frame}[fragile]{\insertsubsection}
\begin{itemize}
\item \cmdbs{includegraphics} komutunu sağlayan \bftt{graphicx} paketini gerektirir.
\item Desteklenen grafik formatları arasında JPEG, PNG ve PDF (genellikle) bulunur.
\end{itemize}
\begin{exampletwouptiny}
\includegraphics[
  width=0.5\textwidth]{gerbil}

\includegraphics[
  width=0.3\textwidth,
  angle=270]{gerbil}
\end{exampletwouptiny}

\tiny{Görüntü lisansı: \href{https://pixabay.com/en/animal-apple-attractive-beautiful-1239390/}{CC0}}
\end{frame}

%%%%%%%%%%%%%%%%%%%%%%%%%%%%%%%%%%%%%%%%%%%%%%%%%%%%%%%%%%%%%%%%%%%%%%%%%%%%%%%
%%%%%%%%%%%%%%%%%%%%%%%%%%%%%%%%%%%%%%%%%%%%%%%%%%%%%%%%%%%%%%%%%%%%%%%%%%%%%%%
%%%%%%%%%%%%%%%%%%%%%%%%%%%%%%%%%%%%%%%%%%%%%%%%%%%%%%%%%%%%%%%%%%%%%%%%%%%%%%%
\begin{frame}[fragile]{Ara: İsteğe Bağlı Argümanlar}
\begin{itemize}
\item İsteğe bağlı argümanlar için süslü parantezler \keystrokebftt{\{} \keystrokebftt{\}} yerine köşeli parantezler \keystrokebftt{[} \keystrokebftt{]} kullanırız.
arguments, instead of braces .

\item \cmdbs{includegraphics} dahil edildiğinde görüntüyü dönüştürmenize izin veren isteğe bağlı bağımsız değişkenleri kabul eder. Örneğin, \bftt{width=0.3\cmdbs{textwidth}} görüntünün çevresindeki metnin 30\%'unu kaplamasına neden olur. (\cmdbs{textwidth}).

\item \cmdbs{documentclass} komutu da isteğe bağlı argümanları kabul eder. Örneğin: 
\mint{latex}{\documentclass[12pt,twocolumn]{article}}
\vskip 3ex metni büyütür (12pt) ve iki sütuna koyar.
\item Bunları nereden öğreniyorsunuz? Daha fazla bilgi için bu sunumun sonundaki slaytlara bakın.
\end{itemize}
\end{frame}

%%%%%%%%%%%%%%%%%%%%%%%%%%%%%%%%%%%%%%%%%%%%%%%%%%%%%%%%%%%%%%%%%%%%%%%%%%%%%%%
%%%%%%%%%%%%%%%%%%%%%%%%%%%%%%%%%%%%%%%%%%%%%%%%%%%%%%%%%%%%%%%%%%%%%%%%%%%%%%%
%%%%%%%%%%%%%%%%%%%%%%%%%%%%%%%%%%%%%%%%%%%%%%%%%%%%%%%%%%%%%%%%%%%%%%%%%%%%%%%
\subsection[fragile]{Floats}
\begin{frame}{\insertsubsection}
\begin{itemize}
\item \LaTeX{}'in şeklin nereye yerleşeceğine karar vermesine izin verin(``yüzebilir'').
\item Ayrıca şekle, \cmdbs{ref} ile başvurulabilecek bir başlık da verebilirsiniz.
\end{itemize}
\begin{minipage}{0.55\linewidth}
\inputminted[fontsize=\scriptsize,frame=single,resetmargins]{latex}%
  {media-graphics.tex}
\end{minipage}
\begin{minipage}{0.35\linewidth}
\includegraphics[width=\textwidth,clip,trim=2in 5in 3in 1in]{media-graphics.pdf}
\end{minipage}

\tiny{Görüntü lisansı: \href{https://pixabay.com/en/animal-apple-attractive-beautiful-1239390/}{CC0}}
\end{frame}

%%%%%%%%%%%%%%%%%%%%%%%%%%%%%%%%%%%%%%%%%%%%%%%%%%%%%%%%%%%%%%%%%%%%%%%%%%%%%%%
%%%%%%%%%%%%%%%%%%%%%%%%%%%%%%%%%%%%%%%%%%%%%%%%%%%%%%%%%%%%%%%%%%%%%%%%%%%%%%%
%%%%%%%%%%%%%%%%%%%%%%%%%%%%%%%%%%%%%%%%%%%%%%%%%%%%%%%%%%%%%%%%%%%%%%%%%%%%%%%
\subsection{Tablolar}
\begin{frame}[fragile]{\insertsubsection}
\begin{itemize}
\item \LaTeX{}'teki tablolara alışmak biraz zaman alır.
\item \bftt{tabularx} paketindeki \bftt{tabular} ortamını kullanın.
\item Argüman, sütun hizalamasını belirtir --- \textbf{l}eft, \textbf{r}ight, \textbf{r}ight.
\begin{exampletwouptiny}
\begin{tabular}{lrr}
Item   & Qty & Unit \$ \\
Widget & 1   & 199.99  \\
Gadget & 2   & 399.99  \\
Cable  & 3   & 19.99   \\
\end{tabular}
\end{exampletwouptiny}
\item Dikey çizgileri de belirtir; yatay çizgiler için \cmdbs{hline} kullanın.
\begin{exampletwouptiny}
\begin{tabular}{|l|r|r|} \hline
Item   & Qty & Unit \$ \\\hline
Widget & 1   & 199.99  \\
Gadget & 2   & 399.99  \\
Cable  & 3   & 19.99   \\\hline
\end{tabular}
\end{exampletwouptiny}
\item Sütunları ayırmak için ampersand \keystrokebftt{\&} işareti ve yeni bir satır başlatmak için çift ters eğik çizgi \keystrokebftt{\bs}\keystrokebftt{\bs} kullanın. (1. bölümde gördüğümüz \bftt{align*} ortamında olduğu gibi).
\end{itemize}
\end{frame}

%%%%%%%%%%%%%%%%%%%%%%%%%%%%%%%%%%%%%%%%%%%%%%%%%%%%%%%%%%%%%%%%%%%%%%%%%%%%%%%
%%%%%%%%%%%%%%%%%%%%%%%%%%%%%%%%%%%%%%%%%%%%%%%%%%%%%%%%%%%%%%%%%%%%%%%%%%%%%%%
%%%%%%%%%%%%%%%%%%%%%%%%%%%%%%%%%%%%%%%%%%%%%%%%%%%%%%%%%%%%%%%%%%%%%%%%%%%%%%%
\addtocontents{toc}{\newpage}
\section{Kaynakçalar}

%%%%%%%%%%%%%%%%%%%%%%%%%%%%%%%%%%%%%%%%%%%%%%%%%%%%%%%%%%%%%%%%%%%%%%%%%%%%%%%
%%%%%%%%%%%%%%%%%%%%%%%%%%%%%%%%%%%%%%%%%%%%%%%%%%%%%%%%%%%%%%%%%%%%%%%%%%%%%%%
%%%%%%%%%%%%%%%%%%%%%%%%%%%%%%%%%%%%%%%%%%%%%%%%%%%%%%%%%%%%%%%%%%%%%%%%%%%%%%%
\begin{frame}{Anahat}
\begin{multicols}{2}
\tableofcontents[currentsection]
\end{multicols}
\end{frame}

%%%%%%%%%%%%%%%%%%%%%%%%%%%%%%%%%%%%%%%%%%%%%%%%%%%%%%%%%%%%%%%%%%%%%%%%%%%%%%%
%%%%%%%%%%%%%%%%%%%%%%%%%%%%%%%%%%%%%%%%%%%%%%%%%%%%%%%%%%%%%%%%%%%%%%%%%%%%%%%
%%%%%%%%%%%%%%%%%%%%%%%%%%%%%%%%%%%%%%%%%%%%%%%%%%%%%%%%%%%%%%%%%%%%%%%%%%%%%%%
\subsection{bib\TeX}
\begin{frame}[fragile]{\insertsubsection{} 1}
\begin{itemize}
\item Referanslarınızı `bibtex' veritabanı biçiminde bir \bftt{.bib} dosyası içine koyun:
\inputminted[fontsize=\scriptsize,frame=single]{latex}{bib-example.bib}
\item Çoğu referans yöneticisi bibtex biçimine aktarabilir.
\end{itemize}
\end{frame}

%%%%%%%%%%%%%%%%%%%%%%%%%%%%%%%%%%%%%%%%%%%%%%%%%%%%%%%%%%%%%%%%%%%%%%%%%%%%%%%
%%%%%%%%%%%%%%%%%%%%%%%%%%%%%%%%%%%%%%%%%%%%%%%%%%%%%%%%%%%%%%%%%%%%%%%%%%%%%%%
%%%%%%%%%%%%%%%%%%%%%%%%%%%%%%%%%%%%%%%%%%%%%%%%%%%%%%%%%%%%%%%%%%%%%%%%%%%%%%%
\begin{frame}[fragile]{\insertsubsection{} 2}
\begin{itemize}
	\item \bftt{.bib} dosyasındaki her girdinin, belgede ona başvurmak için kullanabileceğiniz bir \emph{anahtarı} vardır. Örneğin, \bftt{Jacobson1999Towards} bu makalenin anahtarıdır:

\begin{minted}[fontsize=\small,frame=single]{latex}
@Article{Jacobson1999Towards,
  author = {Van Jacobson},
  ...
}
\end{minted}
\item Ada, yıla ve başlığa bağlı olarak bir anahtar kullanmak iyi bir fikirdir.
\item \LaTeX{} metin içi alıntılarınızı otomatik olarak biçimlendirebilir ve bir referans listesi oluşturabilir; çoğu standart stili bilir ve kendinizinkini tasarlayabilirsiniz.
\end{itemize}
\end{frame}

%%%%%%%%%%%%%%%%%%%%%%%%%%%%%%%%%%%%%%%%%%%%%%%%%%%%%%%%%%%%%%%%%%%%%%%%%%%%%%%
%%%%%%%%%%%%%%%%%%%%%%%%%%%%%%%%%%%%%%%%%%%%%%%%%%%%%%%%%%%%%%%%%%%%%%%%%%%%%%%
%%%%%%%%%%%%%%%%%%%%%%%%%%%%%%%%%%%%%%%%%%%%%%%%%%%%%%%%%%%%%%%%%%%%%%%%%%%%%%%
\begin{frame}[fragile]{\insertsubsection{} 3}
\begin{itemize}
	\item \cmdbs{citep} ve \cmdbs{citet} ile \bftt{natbib} paketini kullanın\footnote{\bftt{biblatex} adında daha fazla özelliğe sahip yeni bir paket var, ancak makale şablonlarının çoğu hala \bftt{natbib} kullanıyor}
\item Sonda \cmdbs{bibliography} ile referans verin ve bir stil, \cmdbs{bibliographystyle}, tanımlayın.

\end{itemize}
\begin{minipage}{0.55\linewidth}
\inputminted[fontsize=\scriptsize,frame=single,resetmargins]{latex}%
  {bib-example.tex}
\end{minipage}
\begin{minipage}{0.35\linewidth}
\includegraphics[width=\textwidth,clip,trim=1.8in 5in 1.8in 1in]{bib-example.pdf}
\end{minipage}
\end{frame}

%%%%%%%%%%%%%%%%%%%%%%%%%%%%%%%%%%%%%%%%%%%%%%%%%%%%%%%%%%%%%%%%%%%%%%%%%%%%%%%
%%%%%%%%%%%%%%%%%%%%%%%%%%%%%%%%%%%%%%%%%%%%%%%%%%%%%%%%%%%%%%%%%%%%%%%%%%%%%%%
%%%%%%%%%%%%%%%%%%%%%%%%%%%%%%%%%%%%%%%%%%%%%%%%%%%%%%%%%%%%%%%%%%%%%%%%%%%%%%%
\subsection{Alıştırma}
\begin{frame}[fragile]{Alıştırma: Hepsini Bir Araya Getirme}

Makaleye önceki alıştırmadan bir resim ve bir kaynakça ekleyin.

\begin{enumerate}
\item Bu örnek dosyaları bilgisayarınıza indirin.

\begin{center}
\fbox{\href{\fileuri/gerbil.jpg?dl=1}{Örnek görüntüyü indirmek için tıklayın}}

\fbox{\href{\fileuri/bib-exercise.bib?dl=1}{Örnek bib dosyasını indirmek için tıklayın}}
\end{center}

\item Overleaf'e yükleyin (proje menüsünü kullanın).

\end{enumerate}
\end{frame}

%%%%%%%%%%%%%%%%%%%%%%%%%%%%%%%%%%%%%%%%%%%%%%%%%%%%%%%%%%%%%%%%%%%%%%%%%%%%%%%
%%%%%%%%%%%%%%%%%%%%%%%%%%%%%%%%%%%%%%%%%%%%%%%%%%%%%%%%%%%%%%%%%%%%%%%%%%%%%%%
%%%%%%%%%%%%%%%%%%%%%%%%%%%%%%%%%%%%%%%%%%%%%%%%%%%%%%%%%%%%%%%%%%%%%%%%%%%%%%%
\section{Sırada Ne Var?}

%%%%%%%%%%%%%%%%%%%%%%%%%%%%%%%%%%%%%%%%%%%%%%%%%%%%%%%%%%%%%%%%%%%%%%%%%%%%%%%
%%%%%%%%%%%%%%%%%%%%%%%%%%%%%%%%%%%%%%%%%%%%%%%%%%%%%%%%%%%%%%%%%%%%%%%%%%%%%%%
%%%%%%%%%%%%%%%%%%%%%%%%%%%%%%%%%%%%%%%%%%%%%%%%%%%%%%%%%%%%%%%%%%%%%%%%%%%%%%%
\begin{frame}{Anahat}
\begin{multicols}{2}
\tableofcontents[currentsection]
\end{multicols}
\end{frame}

%%%%%%%%%%%%%%%%%%%%%%%%%%%%%%%%%%%%%%%%%%%%%%%%%%%%%%%%%%%%%%%%%%%%%%%%%%%%%%%
%%%%%%%%%%%%%%%%%%%%%%%%%%%%%%%%%%%%%%%%%%%%%%%%%%%%%%%%%%%%%%%%%%%%%%%%%%%%%%%
%%%%%%%%%%%%%%%%%%%%%%%%%%%%%%%%%%%%%%%%%%%%%%%%%%%%%%%%%%%%%%%%%%%%%%%%%%%%%%%
\subsection{Daha Güzel Şeyler}
\begin{frame}[fragile]{\insertsubsection}
\begin{itemize}
\item \cmdbs{section} komutlarından bir içindekiler tablosu oluşturmak için \cmdbs{tableofcontents} komutunu ekleyin.

\item \cmdbs{documentclass} özelliğini
\mint{latex}!\documentclass{scrartcl}!
veya
\mint{latex}!\documentclass[12pt]{IEEEtran}! 
olarak değiştirin.

\item Karmaşık bir denklem için kendi komutunuzu tanımlayın:
\begin{exampletwouptiny}
\newcommand{\rperf}{%
  \rho_{\text{perf}}}
$$
\rperf = {\bf c}'{\bf X} + \varepsilon
$$
\end{exampletwouptiny}
\end{itemize}
\end{frame}

%%%%%%%%%%%%%%%%%%%%%%%%%%%%%%%%%%%%%%%%%%%%%%%%%%%%%%%%%%%%%%%%%%%%%%%%%%%%%%%
%%%%%%%%%%%%%%%%%%%%%%%%%%%%%%%%%%%%%%%%%%%%%%%%%%%%%%%%%%%%%%%%%%%%%%%%%%%%%%%
%%%%%%%%%%%%%%%%%%%%%%%%%%%%%%%%%%%%%%%%%%%%%%%%%%%%%%%%%%%%%%%%%%%%%%%%%%%%%%%
\subsection{Daha Güzel Paketler}
\begin{frame}{\insertsubsection}
\begin{itemize}
\item \bftt{beamer}: sunumlar için(bunun gibi!)
\item \bftt{todonotes}: yorumlar ve yapılacaklar yönetimi
\item \bftt{tikz}: harika grafikler oluşturun
\item \bftt{pgfplots}: \LaTeX'de çizgeler oluşturun
\item \bftt{listings}: \LaTeX için kaynak kod yazdırıcı
\item \bftt{spreadtab}: \LaTeX içinde elektronik tablolar oluşturun
\item \bftt{gchords}, \bftt{guitar}: gitar akorları ve tabları
\item \bftt{cwpuzzle}: çapraz bulmacalar
\end{itemize}
Bu paketlerin örnekleri(çoğu) için \url{https://www.overleaf.com/latex/examples} ve \url{http://texample.net} adreslerine bakın.
\end{frame}

%%%%%%%%%%%%%%%%%%%%%%%%%%%%%%%%%%%%%%%%%%%%%%%%%%%%%%%%%%%%%%%%%%%%%%%%%%%%%%%
%%%%%%%%%%%%%%%%%%%%%%%%%%%%%%%%%%%%%%%%%%%%%%%%%%%%%%%%%%%%%%%%%%%%%%%%%%%%%%%
%%%%%%%%%%%%%%%%%%%%%%%%%%%%%%%%%%%%%%%%%%%%%%%%%%%%%%%%%%%%%%%%%%%%%%%%%%%%%%%
\subsection{\LaTeX{} Kurulumu}
\begin{frame}{\insertsubsection}
\begin{itemize}
\item \LaTeX{}'i kendi bilgisayarınızda çalıştırmak için bir \LaTeX{}
\emph{dağıtımı} kullanmalısınız. Bir dağıtım \bftt{latex} programı ve binlerce paket(tipik olarak) içerir.

\begin{itemize}
\item Windows için: \href{http://miktex.org/}{Mik\TeX} veya \href{http://tug.org/texlive/}{\TeX Live}
\item Linux için: \href{http://tug.org/texlive/}{\TeX Live}
\item Mac için: \href{http://tug.org/mactex/}{Mac\TeX}
\end{itemize}
\item Ayrıca \LaTeX{} destekli bir metin editörü de isteyebilirsiniz. Birçok seçenek listesi için \url{http://en.wikipedia.org/wiki/Comparison_of_TeX_editors} adresine bakın.
\item Ayrıca \bftt{latex} ve ilgili araçların nasıl çalıştığı hakkında daha fazla bilgi sahibi olmanız gerekecek - bir sonraki slaytta bulunan kaynaklara bakın.
\end{itemize}
\end{frame}

%%%%%%%%%%%%%%%%%%%%%%%%%%%%%%%%%%%%%%%%%%%%%%%%%%%%%%%%%%%%%%%%%%%%%%%%%%%%%%%
%%%%%%%%%%%%%%%%%%%%%%%%%%%%%%%%%%%%%%%%%%%%%%%%%%%%%%%%%%%%%%%%%%%%%%%%%%%%%%%
%%%%%%%%%%%%%%%%%%%%%%%%%%%%%%%%%%%%%%%%%%%%%%%%%%%%%%%%%%%%%%%%%%%%%%%%%%%%%%%
\subsection{Çevrimiçi Kaynaklar}
\begin{frame}{\insertsubsection}
\begin{itemize}
\item \href{http://en.wikibooks.org/wiki/LaTeX}{The \LaTeX{} Wikibook} ---
mükemmel öğreticiler ve referans materyali.
\item \href{http://tex.stackexchange.com/}{\TeX{} Stack Exchange} --- sorular sorun ve inanılmaz hızlı bir şekilde mükemmel yanıtlar alın
\item \href{http://www.latex-community.org/}{\LaTeX{} Community} --- büyük bir çevrimiçi forum
\item \href{http://ctan.org/}{Comprehensive \TeX{} Archive Network (CTAN)} ---
dört binden fazla paket artı dokümantasyon
\item Google sizi genellikle yukarıdakilerden birine götürecektir.
\end{itemize}
\end{frame}

%%%%%%%%%%%%%%%%%%%%%%%%%%%%%%%%%%%%%%%%%%%%%%%%%%%%%%%%%%%%%%%%%%%%%%%%%%%%%%%
%%%%%%%%%%%%%%%%%%%%%%%%%%%%%%%%%%%%%%%%%%%%%%%%%%%%%%%%%%%%%%%%%%%%%%%%%%%%%%%
%%%%%%%%%%%%%%%%%%%%%%%%%%%%%%%%%%%%%%%%%%%%%%%%%%%%%%%%%%%%%%%%%%%%%%%%%%%%%%%
\begin{frame}
\begin{center}
Teşekkürler ve mutlu \TeX{}lemeler!
\end{center}
\end{frame}

\end{document}

% -- latex understands words, sentences and paragraphs

Words are separated by one or more spaces.  Paragraphs are separated by
one or more blank lines.  The output is not affected by adding extra
spaces or extra blank lines to the input file.

Double quotes are typed like this: ``quoted text''.
Single quotes are typed like this: `single-quoted text'.

Emphasized text is typed like this: \emph{this is emphasized}.
Bold       text is typed like this: \textbf{this is bold}.

-- Adding structure to your document

\section{Hello}

\subsection{World}

\subsection{Foo}

\subsubsection*{Stuff} % star form

\subsubsection*{Results}

-- Labels and cross-references

\label{sec:intro}
\label{sec:method}
\ref{sec:method}

--> maybe introduce the prettyref package here.

-- Mathematics

Inline mathematics: $x + y < 7$.

'Displayed' mathematics:
\begin{equation}
\end{equation}

\begin{equation*}
\end{equation*}

\begin{align}
\end{align}

-- Figures

- Need the graphicx package.

- here we can start introducing options

\includegraphics[width=\textwidth]{}

- where do you find out about these options? --> link to the Wikibook

-- Floating Figures

\begin{figure}
\includegraphics{...}
\caption{\label{}Here is a caption.}
\end{figure}

-- Tables

- not the nicest part of LaTeX

\usepackage{tabularx}

\begin{tabular}{llr}
Item & Quantity & Price (\$) & Amount
Widget & 1 &
\end{tabular}

Bonus points: check out the fp package and the spreadtab package.

-- Document Classes

a .cls file

article

some journal templates come with one

-- Bibliographies



-- For Typesetting Geeks

- dashes: -, --, ---

- ellipsis.

- controlling spaces: ~, \ , \,, \@

- spacing after periods (et al., etc.)

- Nested quotation marks: ``\,`
\vskip 2ex
\item Use the \emph{star form} to display an equation without a number.
\begin{exampletwouptiny}
\begin{equation*}
F(x) = \int_{a}^{x}{f(t) dt}
\end{equation*}
\end{exampletwouptiny}

\begin{itemize}
\item \bftt{equation} and \bftt{equation*} are called \emph{environments}.
\begin{itemize}
  \item The \cmdbs{begin} and \cmdbs{end} commands define the environment.
  \item The \cmd{\$} also starts and ends an environment.
  \item Some commands are defined only within certain environments.
  \item Some commands behave differently in different environments.
\end{itemize}
\end{itemize}
\end{block}
\begin{center}
\fbox{\href{http://ctan.org/}{The Comprehensive \TeX Archive Network (CTAN)}}
\end{center}

%%%%%%%%%%%%%%%%%%%%%%%%%%%%%%%%%%%%%%%%%%%%%%%%%%%%%%%%%%%%%%%%%%%%%%%%%%%%%%%
%%%%%%%%%%%%%%%%%%%%%%%%%%%%%%%%%%%%%%%%%%%%%%%%%%%%%%%%%%%%%%%%%%%%%%%%%%%%%%%
%%%%%%%%%%%%%%%%%%%%%%%%%%%%%%%%%%%%%%%%%%%%%%%%%%%%%%%%%%%%%%%%%%%%%%%%%%%%%%%
\subsection{Typography tweaks}
\begin{frame}{\insertsubsection}
\begin{tabular}{lll}
& character name & used mainly for \ldots \\\hline
\bftt{\bs} & backslash                 & commands, tables \\
\bftt{\{}  & open brace                & commands \\
\bftt{\}}  & close brace               & commands \\
\bftt{\%}  & percent sign              & comments \\
\bftt{\#}  & hash (pound / sharp) sign & custom commands \\
\bftt{\$}  & dollar sign               & equations \\
\bftt{\_}  & underscore                & equations (subscripts) \\
\bftt{\^}  & caret                     & equations (superscripts) \\
\bftt{\&}  & ampersand                 & tables \\
\bftt{\~}  & tilde                     & spacing \\
\end{tabular}
\end{frame}

%\item We've used several environments:
%\vskip 1ex
%{\scriptsize
%\begin{tabular}{ll}
%\cmdbs{begin}\bftt{\{document\}}\ldots\cmdbs{end}\bftt{\{document\}} &
%  document environment \\
%\cmdbs{begin}\bftt{\{itemize\}}\ldots\cmdbs{end}\bftt{\{itemize\}} &
%  itemized list environment \\
%\bftt{\$\ldots\$}     & \emph{in-text} math environment \\
%\bftt{\$\$\ldots\$\$} & \emph{displayed} math environment \\
%\cmdbs{begin}\bftt{\{equation\}}\ldots\cmdbs{end}\bftt{\{equation\}} &
%  displayed math environment w/ number
%\end{tabular}
%}
