\documentclass[aspectratio=169]{beamer}

\input{preamble.tex}

\subtitle{Bölüm 1: Temeller}

\begin{document}

%%%%%%%%%%%%%%%%%%%%%%%%%%%%%%%%%%%%%%%%%%%%%%%%%%%%%%%%%%%%%%%%%%%%%%%%%%%%%%%
%%%%%%%%%%%%%%%%%%%%%%%%%%%%%%%%%%%%%%%%%%%%%%%%%%%%%%%%%%%%%%%%%%%%%%%%%%%%%%%
%%%%%%%%%%%%%%%%%%%%%%%%%%%%%%%%%%%%%%%%%%%%%%%%%%%%%%%%%%%%%%%%%%%%%%%%%%%%%%%
\begin{frame}
\titlepage
\end{frame}

%%%%%%%%%%%%%%%%%%%%%%%%%%%%%%%%%%%%%%%%%%%%%%%%%%%%%%%%%%%%%%%%%%%%%%%%%%%%%%%
%%%%%%%%%%%%%%%%%%%%%%%%%%%%%%%%%%%%%%%%%%%%%%%%%%%%%%%%%%%%%%%%%%%%%%%%%%%%%%%
%%%%%%%%%%%%%%%%%%%%%%%%%%%%%%%%%%%%%%%%%%%%%%%%%%%%%%%%%%%%%%%%%%%%%%%%%%%%%%%
\begin{frame}{Neden \LaTeX{}?}
\begin{itemize}
\item Güzel dokümanlar oluşturur
\begin{itemize}
\item Özellikle matematiksel ifadeler için
\end{itemize}
%
\item Bilim adamları tarafından, bilim adamları için geliştirilmiştir
\begin{itemize}
\item Büyük ve aktif bir topluluğa sahiptir
\end{itemize}
%
\item Güçlüdür --- genişletilebilir
\begin{itemize}
\item Makaleler, sunumlar, hesap çizelgeleri, \ldots vb. için paketler içerir
\end{itemize}
\end{itemize}
\end{frame}

%%%%%%%%%%%%%%%%%%%%%%%%%%%%%%%%%%%%%%%%%%%%%%%%%%%%%%%%%%%%%%%%%%%%%%%%%%%%%%%
%%%%%%%%%%%%%%%%%%%%%%%%%%%%%%%%%%%%%%%%%%%%%%%%%%%%%%%%%%%%%%%%%%%%%%%%%%%%%%%
%%%%%%%%%%%%%%%%%%%%%%%%%%%%%%%%%%%%%%%%%%%%%%%%%%%%%%%%%%%%%%%%%%%%%%%%%%%%%%%
\begin{frame}[fragile]{Nasıl çalışır?}
\begin{itemize}
%\item You write your document in \texttt{plain text} with \cmd{commands} that describe its structure and meaning.
%\item The \texttt{latex} program processes your text and commands to produce a beautifully formatted document.
\item Dokümanınızı \texttt{düz metin} olarak \cmd{komutlarla} yazarsınız. Komutlar metnin yapısını ve anlamını tanımlar.
\item \texttt{Latex} programı yazdığınız metin ve komutları işleyerek güzel biçimlendirilmiş dokümanlar üretir.
\end{itemize}
\vskip 2ex
\begin{center}
%The rain in Spain falls \emph{mainly} on the plain.
\begin{minted}[frame=single]{latex}
İspanya'da yağmur \emph{çoğunlukla} ovaya yağar.
\end{minted}
\vskip 2ex
\tikz\node[single arrow,fill=gray,font=\ttfamily\bfseries,%
  rotate=270,xshift=-1em]{latex};
\vskip 2ex
\fbox{İspanya'da yağmur \emph{çoğunlukla} ovaya yağar.}
\end{center}
\end{frame}

%%%%%%%%%%%%%%%%%%%%%%%%%%%%%%%%%%%%%%%%%%%%%%%%%%%%%%%%%%%%%%%%%%%%%%%%%%%%%%%
%%%%%%%%%%%%%%%%%%%%%%%%%%%%%%%%%%%%%%%%%%%%%%%%%%%%%%%%%%%%%%%%%%%%%%%%%%%%%%%
%%%%%%%%%%%%%%%%%%%%%%%%%%%%%%%%%%%%%%%%%%%%%%%%%%%%%%%%%%%%%%%%%%%%%%%%%%%%%%%
\begin{frame}[fragile]{Komutlar ve çıktıları için daha fazla örnek\ldots}
\begin{exampletwoup}
\begin{itemize}
\item Ekmek
\item Su
\item Kahve
\end{itemize}
\end{exampletwoup}
\vskip 2ex
\begin{exampletwoup}
\begin{figure}
\includegraphics{gerbil}
\end{figure}
\end{exampletwoup}
\vskip 2ex
\begin{exampletwoup}
\begin{equation}
\alpha + \beta + 1
\end{equation}
\end{exampletwoup}

\tiny{Görüntü lisansı: \href{https://pixabay.com/en/animal-apple-attractive-beautiful-1239390/}{CC0}}
\end{frame}

%%%%%%%%%%%%%%%%%%%%%%%%%%%%%%%%%%%%%%%%%%%%%%%%%%%%%%%%%%%%%%%%%%%%%%%%%%%%%%%
%%%%%%%%%%%%%%%%%%%%%%%%%%%%%%%%%%%%%%%%%%%%%%%%%%%%%%%%%%%%%%%%%%%%%%%%%%%%%%%
%%%%%%%%%%%%%%%%%%%%%%%%%%%%%%%%%%%%%%%%%%%%%%%%%%%%%%%%%%%%%%%%%%%%%%%%%%%%%%%
\begin{frame}[fragile]{Davranış ayarlaması}

\begin{itemize}
\item Komutları kullanarak `ne olduğunu' tanımlayın, `nasıl gözüktüğünü' değil%Use commands to describe `what it is', not `how it looks'.
\item İçeriğinize odaklanın.%Focus on your content.
\item \LaTeX{}'in işini yapmasına izin verin.%Let \LaTeX{} do its job.
\end{itemize}
\end{frame}

%%%%%%%%%%%%%%%%%%%%%%%%%%%%%%%%%%%%%%%%%%%%%%%%%%%%%%%%%%%%%%%%%%%%%%%%%%%%%%%
%%%%%%%%%%%%%%%%%%%%%%%%%%%%%%%%%%%%%%%%%%%%%%%%%%%%%%%%%%%%%%%%%%%%%%%%%%%%%%%
%%%%%%%%%%%%%%%%%%%%%%%%%%%%%%%%%%%%%%%%%%%%%%%%%%%%%%%%%%%%%%%%%%%%%%%%%%%%%%%
\section{Temeller}

%%%%%%%%%%%%%%%%%%%%%%%%%%%%%%%%%%%%%%%%%%%%%%%%%%%%%%%%%%%%%%%%%%%%%%%%%%%%%%%
%%%%%%%%%%%%%%%%%%%%%%%%%%%%%%%%%%%%%%%%%%%%%%%%%%%%%%%%%%%%%%%%%%%%%%%%%%%%%%%
%%%%%%%%%%%%%%%%%%%%%%%%%%%%%%%%%%%%%%%%%%%%%%%%%%%%%%%%%%%%%%%%%%%%%%%%%%%%%%%
\subsection{Başlangıç}
\begin{frame}[fragile]{\insertsubsection}
\begin{itemize}
\item Asgari bir \LaTeX{} dokümanı:
\inputminted[frame=single]{latex}{basics.tex}
\item Komutlar \emph{ters eğik çizgi} ile başlar \keystrokebftt{\bs}.
\item Her doküman bir \cmdbs{documentclass} komutu ile başlar.
\item Süslü parantezler \keystrokebftt{\{} \keystrokebftt{\}} içindeki \emph{argümanlar}  \LaTeX{}'e ne türde bir doküman oluşturduğumuzu söyler: bir makale.
\item Yüzde sembolü \keystrokebftt{\%} bir \emph{açıklama} başlatır --- \LaTeX{}
satırın kalanını göz ardı eder.
\end{itemize}
\end{frame}

%%%%%%%%%%%%%%%%%%%%%%%%%%%%%%%%%%%%%%%%%%%%%%%%%%%%%%%%%%%%%%%%%%%%%%%%%%%%%%%
%%%%%%%%%%%%%%%%%%%%%%%%%%%%%%%%%%%%%%%%%%%%%%%%%%%%%%%%%%%%%%%%%%%%%%%%%%%%%%%
%%%%%%%%%%%%%%%%%%%%%%%%%%%%%%%%%%%%%%%%%%%%%%%%%%%%%%%%%%%%%%%%%%%%%%%%%%%%%%%
\begin{frame}[fragile]{\wllogo{} ile \insertsubsection{}}
\begin{itemize}
\item Overleaf \LaTeX{}  kullanarak doküman oluşturmayı sağlayan bir web sitesidir.
\item Size sonuçları göstermek için \LaTeX{} dokümanınızı otomatik olarak `derler'.
\vskip 2em
\begin{center}
\fbox{\href{\wlnewdoc{basics.tex}}{%
%Click here to open the example document in \wllogo{} \\
Örnek dokümanı \wllogo{} içinde açmak için buraya tıklayın
}}
\\[1ex]\scriptsize{}
En iyi sonucu almak için \href{http://www.google.com/chrome}{Google Chrome} ya da güncel \href{http://www.mozilla.org/en-US/firefox/new/}{FireFox} kullanın.
\end{center}
\vskip 2ex
\item Yansıları geçtikçe, örnekleri Overleaf üzerindeki örnek dokümana yazarak deneyin.
\item \textbf{Biz ilerlerken bunu gerçekten yapmalısınız!}
\end{itemize}
\end{frame}

%%%%%%%%%%%%%%%%%%%%%%%%%%%%%%%%%%%%%%%%%%%%%%%%%%%%%%%%%%%%%%%%%%%%%%%%%%%%%%%
%%%%%%%%%%%%%%%%%%%%%%%%%%%%%%%%%%%%%%%%%%%%%%%%%%%%%%%%%%%%%%%%%%%%%%%%%%%%%%%
%%%%%%%%%%%%%%%%%%%%%%%%%%%%%%%%%%%%%%%%%%%%%%%%%%%%%%%%%%%%%%%%%%%%%%%%%%%%%%%
\subsection{Metin Dizgileme}
\begin{frame}[fragile]{\insertsubsection{}}
\small
\begin{itemize}
\item Metninizi \cmdbegin{document} ile \cmdend{document} arasına yazın.
\item Çoğunlukla, metninizi normal olarak yazabilirsiniz.
\begin{exampletwouptiny}
Kelimeler bir veya daha fazla bosluk
ile birbirinden ayrilir.

Paragraflar bir veya daha fazla bos 
satir ile birbirinden ayrilir.
\end{exampletwouptiny}
\item Kaynak dosyadaki boşlukar çıktıda kaybolur.
\begin{exampletwouptiny}
Ispanya'da  yagmur    cogunlukla
ovalara    yagar.
\end{exampletwouptiny}
\end{itemize}
\end{frame}

%%%%%%%%%%%%%%%%%%%%%%%%%%%%%%%%%%%%%%%%%%%%%%%%%%%%%%%%%%%%%%%%%%%%%%%%%%%%%%%
%%%%%%%%%%%%%%%%%%%%%%%%%%%%%%%%%%%%%%%%%%%%%%%%%%%%%%%%%%%%%%%%%%%%%%%%%%%%%%%
%%%%%%%%%%%%%%%%%%%%%%%%%%%%%%%%%%%%%%%%%%%%%%%%%%%%%%%%%%%%%%%%%%%%%%%%%%%%%%%
\begin{frame}[fragile]{\insertsubsection{}: Uyarılar}
\small
\begin{itemize}
\item Tırnak işaretleri biraz zordur:\\
Solda bir ters kesme \keystroke{\`{}} ve sağda kesme işareti \keystroke{\'{}} kullanın.
\begin{exampletwouptiny}
Tek tirnak: `metin'.

Cift tirnak: ``metin''.
\end{exampletwouptiny}

\item \LaTeX'de bazı yaygın olarak kullanılan karekterlerin özel anlamları vardır:\\[1ex]
\begin{tabular}{cl}
\keystrokebftt{\%} & yüzde işareti              \\
\keystrokebftt{\#} & diyez işareti \\
\keystrokebftt{\&} & ve işareti                 \\
\keystrokebftt{\$} & dolar işareti               \\
\end{tabular}
\item Sadece bunları yazarsanız hata alırsınız. Eğer herhangi birinin çıktıda yer almasını istiyorsanız karakterin öncesinde ters eğik çizgi kullanın.
\begin{exampletwoup}
\$\%\&\#!
\end{exampletwoup}
\end{itemize}
\end{frame}

%%%%%%%%%%%%%%%%%%%%%%%%%%%%%%%%%%%%%%%%%%%%%%%%%%%%%%%%%%%%%%%%%%%%%%%%%%%%%%%
%%%%%%%%%%%%%%%%%%%%%%%%%%%%%%%%%%%%%%%%%%%%%%%%%%%%%%%%%%%%%%%%%%%%%%%%%%%%%%%
%%%%%%%%%%%%%%%%%%%%%%%%%%%%%%%%%%%%%%%%%%%%%%%%%%%%%%%%%%%%%%%%%%%%%%%%%%%%%%%
\begin{frame}[fragile]{Hataları İşleme}
\begin{itemize}
\item Dokümanınızı derlerken \LaTeX{}'in kafası karışabilir. Böyle bir durumda, çıktı oluşmadan önce düzeltmeniz gereken, bir hata ile sonlanır.
\item Örneğin \cmdbs{emph} komutunu \cmdbs{meph} şeklinde yanlış yazarsanız, \LaTeX{} ``undefined control sequence'' hatası ile sonlanır.
Çünkü ``meph'' \LaTeX{}'in tanıdığı komutlardan birisi değildir.
\end{itemize}
\begin{block}{Hata Tavsiyesi}
\begin{enumerate}
\item Paniklemeyin! Hatalar olacaktır.
\item Hatalar ortaya çıkar çıkmaz onları düzeltin --- eğer yazdığınız bir şey hataya neden olduysa hata ayıklamaya oradan başlayabilirsiniz.
\item Eğer birden fazla hata varsa, ilk hata ile başlayın --- sebebi daha yukarda bile olabilir. 
\end{enumerate}
\end{block}
\end{frame}

%%%%%%%%%%%%%%%%%%%%%%%%%%%%%%%%%%%%%%%%%%%%%%%%%%%%%%%%%%%%%%%%%%%%%%%%%%%%%%%
%%%%%%%%%%%%%%%%%%%%%%%%%%%%%%%%%%%%%%%%%%%%%%%%%%%%%%%%%%%%%%%%%%%%%%%%%%%%%%%
%%%%%%%%%%%%%%%%%%%%%%%%%%%%%%%%%%%%%%%%%%%%%%%%%%%%%%%%%%%%%%%%%%%%%%%%%%%%%%%
\begin{frame}[fragile]{Dizgi Egzersizi 1}

\begin{block}{Bunu \LaTeX'de yazın:
\footnote{\url{http://en.wikipedia.org/wiki/Economy_of_the_United_States}}}
In March 2006, Congress raised that ceiling an additional \$0.79 trillion to
\$8.97 trillion, which is approximately 68\% of GDP. As of October 4, 2008, the
``Emergency Economic Stabilization Act of 2008'' raised the current debt ceiling
to \$11.3 trillion.
\end{block}
\vskip 2ex
\begin{center}
\fbox{\href{\wlnewdoc{basics-exercise-1.tex}}{%
Bu egzersizi \wllogo{} ile açmak için tıklayın.}}
\end{center}

\begin{itemize}
\item İpucu: karakterlerin özel anlamına dikkat edin!
\item Denedikten sonra,
\fbox{\href{\wlnewdoc{basics-exercise-1-solution.tex}}{%
benim çözümümü görmek için tıklayın}}.
\end{itemize}
\end{frame}

%%%%%%%%%%%%%%%%%%%%%%%%%%%%%%%%%%%%%%%%%%%%%%%%%%%%%%%%%%%%%%%%%%%%%%%%%%%%%%%
%%%%%%%%%%%%%%%%%%%%%%%%%%%%%%%%%%%%%%%%%%%%%%%%%%%%%%%%%%%%%%%%%%%%%%%%%%%%%%%
%%%%%%%%%%%%%%%%%%%%%%%%%%%%%%%%%%%%%%%%%%%%%%%%%%%%%%%%%%%%%%%%%%%%%%%%%%%%%%%
\subsection{Matematik Dizgileme}
\begin{frame}[fragile]{\insertsubsection{}: Dolar İşaretleri}
\begin{itemize}
\item Dolar işaretleri \keystrokebftt{\$} neden özeldir? Metnin içinde matematik ifadeleri işaretlemek için kullanılır.\\[1ex]
\begin{exampletwouptiny}
% iyi degil:
a ve b iki ayri pozitif tamsayi olsun
ve c = a - b + 1 olsun.

% daha iyi:
$a$ ve $b$ iki ayri pozitif tamsayi 
ve $c = a - b + 1$ olsun.
\end{exampletwouptiny}
\item Dolar işaretlerini her zaman çift olarak kullanın --- birisi matematik ifadesini başlatmak için, diğeri bitirmek için.
\item \LaTeX{} boşlukarı otomatik olarak işler; sizin boşluklarınızı göz ardı eder.

\begin{exampletwouptiny}
$y=mx+b$ olsun \ldots

$y = m x + b$ olsun \ldots
\end{exampletwouptiny}
\end{itemize}
\end{frame}

%%%%%%%%%%%%%%%%%%%%%%%%%%%%%%%%%%%%%%%%%%%%%%%%%%%%%%%%%%%%%%%%%%%%%%%%%%%%%%%
%%%%%%%%%%%%%%%%%%%%%%%%%%%%%%%%%%%%%%%%%%%%%%%%%%%%%%%%%%%%%%%%%%%%%%%%%%%%%%%
%%%%%%%%%%%%%%%%%%%%%%%%%%%%%%%%%%%%%%%%%%%%%%%%%%%%%%%%%%%%%%%%%%%%%%%%%%%%%%%
\begin{frame}[fragile]{\insertsubsection{}: Notasyon}
\begin{itemize}
\item Üst simgeler için şapka işareti \keystrokebftt{\^}, alt simgeler için alt çizgi \keystrokebftt{\_} kullanın.
\begin{exampletwouptiny}
$y = c_2 x^2 + c_1 x + c_0$
\end{exampletwouptiny}
\vskip 2ex

\item Alt ve üst simgeleri gruplamak için süslü parantezleri \keystrokebftt{\{} \keystrokebftt{\}} kullanın.
\begin{exampletwouptiny}
$F_n = F_n-1 + F_n-2$     % oops!

$F_n = F_{n-1} + F_{n-2}$ % ok!
\end{exampletwouptiny}
\vskip 2ex

\item Yunan harfleri ve yaygın gösterimler için komutlar vardır.
\begin{exampletwouptiny}
$\mu = A e^{Q/RT}$

$\Omega = \sum_{k=1}^{n} \omega_k$
\end{exampletwouptiny}
\end{itemize}
\end{frame}

%%%%%%%%%%%%%%%%%%%%%%%%%%%%%%%%%%%%%%%%%%%%%%%%%%%%%%%%%%%%%%%%%%%%%%%%%%%%%%%
%%%%%%%%%%%%%%%%%%%%%%%%%%%%%%%%%%%%%%%%%%%%%%%%%%%%%%%%%%%%%%%%%%%%%%%%%%%%%%%
%%%%%%%%%%%%%%%%%%%%%%%%%%%%%%%%%%%%%%%%%%%%%%%%%%%%%%%%%%%%%%%%%%%%%%%%%%%%%%%
\begin{frame}[fragile]{\insertsubsection{}: Görüntülenen Denklemler}
\begin{itemize}
\item Eğer denklem büyük ve korkutucu  ise kendi satırında görüntülemek için \cmdbegin{equation} ve \cmdend{equation} arasına yazın.\\[2ex]
\begin{exampletwouptiny}
Ikinci dereceden bir denklemin
kokleri
\begin{equation}
x = \frac{-b \pm \sqrt{b^2 - 4ac}}
         {2a}
\end{equation}
ile hesaplanir. $a$, $b$ ve $c$ \ldots
\end{exampletwouptiny}
\vskip 1em
{\scriptsize Dikkat: \LaTeX{} çoğunlukla matematik ifadelerindeki boşlukları göz ardı eder fakat denklemdeki boş satırları işleyemez. --- matematiksel ifadelerde boş satır kullanmayın.}
\end{itemize}
\end{frame}

%%%%%%%%%%%%%%%%%%%%%%%%%%%%%%%%%%%%%%%%%%%%%%%%%%%%%%%%%%%%%%%%%%%%%%%%%%%%%%%
%%%%%%%%%%%%%%%%%%%%%%%%%%%%%%%%%%%%%%%%%%%%%%%%%%%%%%%%%%%%%%%%%%%%%%%%%%%%%%%
%%%%%%%%%%%%%%%%%%%%%%%%%%%%%%%%%%%%%%%%%%%%%%%%%%%%%%%%%%%%%%%%%%%%%%%%%%%%%%%
\begin{frame}[fragile]{Perde Arkası: Ortamlar}
\begin{itemize}
\item \bftt{equation} bir \emph{ortamdır} --- bir içerik.
\item Bir komut farklı bağlamlarda farklı çıktılar üretebilir.
\begin{exampletwouptiny}
Metnin icinde 
$ \Omega = \sum_{k=1}^{n} \omega_k $
yazabiliriz, ya da goruntulemek icin
\begin{equation}
  \Omega = \sum_{k=1}^{n} \omega_k
\end{equation}
yazabiliriz.
\end{exampletwouptiny}
\vskip 2ex
\item $\Sigma$ sembolünün \bftt{equation} ortamında nasıl daha büyük olduğuna ve aynı komut kullanılsa bile alt ve üst simgelerin konumunun nasıl değiştiğine dikkat edin.
\vskip 1em
{\scriptsize Aslında, \bftt{\$...\$} ifadesini 
\cmdbegin{math}\bftt{...}\cmdend{math} şeklinde de yazabilirdik.}
\end{itemize}
\end{frame}

%%%%%%%%%%%%%%%%%%%%%%%%%%%%%%%%%%%%%%%%%%%%%%%%%%%%%%%%%%%%%%%%%%%%%%%%%%%%%%%
%%%%%%%%%%%%%%%%%%%%%%%%%%%%%%%%%%%%%%%%%%%%%%%%%%%%%%%%%%%%%%%%%%%%%%%%%%%%%%%
%%%%%%%%%%%%%%%%%%%%%%%%%%%%%%%%%%%%%%%%%%%%%%%%%%%%%%%%%%%%%%%%%%%%%%%%%%%%%%%
\begin{frame}[fragile]{Perde Arkası: Ortamlar}
\begin{itemize}
\item \cmdbs{begin} ve \cmdbs{end} komutları birçok farklı ortam oluşturmak için kullanılır.
\vskip 2ex

\item \bftt{itemize} ve \bftt{enumerate} ortamları liste oluştururlar.
\begin{exampletwouptiny}
\begin{itemize} % madde imleri icin
\item Ekmek
\item Su
\end{itemize}

\begin{enumerate} % numaralandirma icin
\item Ekmek
\item Su
\end{enumerate}
\end{exampletwouptiny}
\end{itemize}
\end{frame}

%%%%%%%%%%%%%%%%%%%%%%%%%%%%%%%%%%%%%%%%%%%%%%%%%%%%%%%%%%%%%%%%%%%%%%%%%%%%%%%
%%%%%%%%%%%%%%%%%%%%%%%%%%%%%%%%%%%%%%%%%%%%%%%%%%%%%%%%%%%%%%%%%%%%%%%%%%%%%%%
%%%%%%%%%%%%%%%%%%%%%%%%%%%%%%%%%%%%%%%%%%%%%%%%%%%%%%%%%%%%%%%%%%%%%%%%%%%%%%%
\begin{frame}[fragile]{Perde Arkası: Paketler}

\begin{itemize}
\item Şu ana kadar kullandığımız bütün komutlar ve ortamlar \LaTeX{} içinde yerleşiktir.

\item \emph{Paketler} ekstra komut ve ortamların bulunduğu kütüphanelerdir. Binlerce ücretsiz paket bulunmaktadır. 
\item Kullanmak istediğimiz paketlerin her birini \emph{başlangıç} kısmında \cmdbs{usepackage} komutu ile yüklemeliyiz.

\item Örnek: Amerikan Matematik Topluluğu için \bftt{amsmath}.
\begin{minted}[fontsize=\small,frame=single]{latex}
\documentclass{article}
\usepackage{amsmath} % başlangıç
\begin{document}
% artık burada amsmath komutlarını kullanabiliriz
\end{document}
\end{minted}
\end{itemize}
\end{frame}

%%%%%%%%%%%%%%%%%%%%%%%%%%%%%%%%%%%%%%%%%%%%%%%%%%%%%%%%%%%%%%%%%%%%%%%%%%%%%%%
%%%%%%%%%%%%%%%%%%%%%%%%%%%%%%%%%%%%%%%%%%%%%%%%%%%%%%%%%%%%%%%%%%%%%%%%%%%%%%%
%%%%%%%%%%%%%%%%%%%%%%%%%%%%%%%%%%%%%%%%%%%%%%%%%%%%%%%%%%%%%%%%%%%%%%%%%%%%%%%
\begin{frame}[fragile]{\insertsubsection{}: \bftt{amsmath} örnekleri}
\begin{itemize}
\item Numaralandırılmayan denklemler için \bftt{equation*} (``equation-yıldızlı'') kullanın.
\begin{exampletwouptiny}
\begin{equation*}
  \Omega = \sum_{k=1}^{n} \omega_k
\end{equation*}
\end{exampletwouptiny}
\item \LaTeX{}, bitişik harfleri değişkenlerin çarpımı şeklinde ele alır, her zaman istediğiniz bir şey değildir. \bftt{amsmath} birçok yaygın matematik operatörü için komutlar tanımlar.
\begin{exampletwouptiny}
\begin{equation*} % kotu!
 min_{x,y} (1-x)^2 + 100(y-x^2)^2
\end{equation*}
\begin{equation*} % iyi!
\min_{x,y}{(1-x)^2 + 100(y-x^2)^2}
\end{equation*}
\end{exampletwouptiny}
\item Diğerleri için \cmdbs{operatorname} kullanabilirsiniz.
\begin{exampletwouptiny}
\begin{equation*}
\beta_i =
\frac{\operatorname{Cov}(R_i, R_m)}
     {\operatorname{Var}(R_m)}
\end{equation*}
\end{exampletwouptiny}
\end{itemize}
\end{frame}

%%%%%%%%%%%%%%%%%%%%%%%%%%%%%%%%%%%%%%%%%%%%%%%%%%%%%%%%%%%%%%%%%%%%%%%%%%%%%%%
%%%%%%%%%%%%%%%%%%%%%%%%%%%%%%%%%%%%%%%%%%%%%%%%%%%%%%%%%%%%%%%%%%%%%%%%%%%%%%%
%%%%%%%%%%%%%%%%%%%%%%%%%%%%%%%%%%%%%%%%%%%%%%%%%%%%%%%%%%%%%%%%%%%%%%%%%%%%%%%
\begin{frame}[fragile]{\insertsubsection{}: \bftt{amsmath} örnekleri}
\begin{itemize}{\small
\item Eşittir işareti ile bir denkem dizisini \bftt{align*} ortamı ilehizalama
\begin{align*}
(x+1)^3 &= (x+1)(x+1)(x+1) \\
        &= (x+1)(x^2 + 2x + 1) \\
        &= x^3 + 3x^2 + 3x + 1
\end{align*}


% for whatever reason, this doesn't play well with the twoup environment
\begin{minted}[fontsize=\small,frame=single]{latex}
\begin{align*}
(x+1)^3 &= (x+1)(x+1)(x+1) \\
        &= (x+1)(x^2 + 2x + 1) \\
        &= x^3 + 3x^2 + 3x + 1
\end{align*}
\end{minted}
\item Bir ampersand \keystrokebftt{\&} sembolü sol sütun($=$'den önce) ile sağ sütunu($=$'den sonra) birbirinden ayırır.
\item Çift ters eğik çizgi \keystrokebftt{\bs}\keystrokebftt{\bs} yeni bir satır başlatır.
}\end{itemize}
\end{frame}


%%%%%%%%%%%%%%%%%%%%%%%%%%%%%%%%%%%%%%%%%%%%%%%%%%%%%%%%%%%%%%%%%%%%%%%%%%%%%%%
%%%%%%%%%%%%%%%%%%%%%%%%%%%%%%%%%%%%%%%%%%%%%%%%%%%%%%%%%%%%%%%%%%%%%%%%%%%%%%%
%%%%%%%%%%%%%%%%%%%%%%%%%%%%%%%%%%%%%%%%%%%%%%%%%%%%%%%%%%%%%%%%%%%%%%%%%%%%%%%
\begin{frame}[fragile]{Dizgi Egzersizi 2}

\begin{block}{\LaTeX'de Dizgi:}
Let $X_1, X_2, \ldots, X_n$ be a sequence of independent and identically
distributed random variables with $\operatorname{E}[X_i] = \mu$ and
$\operatorname{Var}[X_i] = \sigma^2 < \infty$, and let
\begin{equation*}
S_n = \frac{1}{n}\sum_{i}^{n} X_i
\end{equation*}
denote their mean. Then as $n$ approaches infinity, the random variables
$\sqrt{n}(S_n - \mu)$ converge in distribution to a normal $N(0, \sigma^2)$.
\end{block}
\vskip 2ex
\begin{center}
\fbox{\href{\wlnewdoc{basics-exercise-2.tex}}{%
\wllogo{}'de açmak için tıklayın}}
\end{center}
\begin{itemize}
\item İpucu: $\infty$ sembolü için \cmdbs{infty} komutunu kullanın.
\item Denedikten sonra,
\fbox{\href{\wlnewdoc{basics-exercise-2-solution.tex}}{%
benim çözümümü görmek için tıklayın}}.
\end{itemize}
\end{frame}

%%%%%%%%%%%%%%%%%%%%%%%%%%%%%%%%%%%%%%%%%%%%%%%%%%%%%%%%%%%%%%%%%%%%%%%%%%%%%%%
%%%%%%%%%%%%%%%%%%%%%%%%%%%%%%%%%%%%%%%%%%%%%%%%%%%%%%%%%%%%%%%%%%%%%%%%%%%%%%%
%%%%%%%%%%%%%%%%%%%%%%%%%%%%%%%%%%%%%%%%%%%%%%%%%%%%%%%%%%%%%%%%%%%%%%%%%%%%%%%
\begin{frame}{1. Bölümün Sonu}
\begin{itemize}
\item Tebrikler! Aşağıdakileri nasıl yapacağınızı öğrendiniz\ldots
\begin{itemize}
\item \LaTeX'de dizgileme.
\item Bir çok farklı komutun kullanımı.
\item Hata ortaya çıkınca işleme.
\item Matematik ifadeleri yazma.
\item Birçok farklı ortamı kullanma.
\item Paketleri yükleme.
\end{itemize}
\item Bu inanılmaz!
\item 2. bölümde \LaTeX kullanarak bölümleri, çapraz referansları, şekilleri, tabloları ve kaynakçası olan yapısal dokümanların nasıl oluşturulacağını göreceğiz. Görüşmek üzere!
\end{itemize}
\end{frame}

\end{document}
